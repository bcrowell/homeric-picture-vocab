\documentclass[12pt,openany]{book}
\usepackage{fontspec}
\setmainfont{DejaVu Serif}
\usepackage[bindingoffset=0.2in,%
            left=0in,right=0.5in,top=1in,bottom=1in,%
            footskip=.25in]{geometry}
\usepackage[inkscapearea=drawing]{svg}
\usepackage{titlesec}

\titleformat{\chapter}[block]
  {\normalfont\huge\bfseries\centering}{}{0em}{\Huge}
\titlespacing*{\chapter}{0pt}{-19pt}{50pt} % left, before, after

\newenvironment{notes}{\vfill\scriptsize\begin{flushright}}{\end{flushright}}
% \tiny would be smaller, \footnotesize larger

\newcommand{\fig}[1]{\begin{center}\includesvg{#1}\end{center}}

%\setsvg{inkscapelatex=false}
% https://tex.stackexchange.com/questions/213432/how-not-to-extract-text-when-using-includesvg

\begin{document}

%---------------------------------------

\chapter{οἵ ἀριθμοί}

\fig{numbers/numbers.svg}

\begin{notes}
Homer has 4=τέσσαρα, other ancient dialects τέτταρα.
Modern: ένα, τέσσαρα, έξι.

Cognates: dual, triangle, pentagon (etc.), decade,
prototype, Deuteronomy, tritium.
\end{notes}

%---------------------------------------

\chapter{τά ὄμματα, τα πρόσωπα}

\fig{face/face.svg}

\begin{notes}
In Homer, face=τά ὄμματα (lit. eyes), τα πρόσωπα (or -πατα) (pl.).
Modern πρόσωπο. Lit. ``next to the eyes.''
``Hair,'' τρίχες, is always plural, so the sing.~nom.~root θρίξ never occurs in Homer.
``Ear'' usually occurs as οὔατα or οὔατος, often in stylized descriptions of actions such as
skinning a man and cutting off his ears with ``unpitying brass,'' ``τοῦ δ' ἀπὸ μὲν ῥῖνάς τε καὶ οὔατα νηλέϊ χαλκῷ'' (Odyssey 22.476).


Cognates: cephalic, stomach,
acoustic, glossary, tresses.
\end{notes}

%---------------------------------------

\chapter{τα γυῖα, τα μέλεα, τό δέμας}

\fig{body/body.svg}

\begin{notes}
Modern: χέρι, πόδι, άνδρας, γυναίκα. In Homer, σῶμα=dead body, τα γυῖα, τα μέλεα=the body (lit. ``the limbs'').
κνήμη = lower leg, shin bone. Homer has several words for ``limb,'' but none for the (whole) leg specifically.
δάκτυλος never occurs as an independent word in Homer, but is found 27 times in the
set phrase ῥοδοδάκτυλος ̓ηώς, rosy-fingered dawn. Cognates: κνήμη=ham; gynecology; pedal; androgynous; somatic; μέλεα=member, melody;
notochord; cardiology; hemoglobin; χρώς=chroma (skin, color, complexion).
\end{notes}

%---------------------------------------

\chapter{τα ζῷα, τα θηρία}

\fig{animals/animals.svg} % gives invalid glyph error

\begin{notes}
κάπρος=boar,  θώς=jackal, Iliad 11.  ἀετός, Iliad 17, 21. Cognates: zoo, canine, hippo, lion, lycanthrope, ornithology.
Modern dog=σκύλος, sheep=πρόβατο. 
\end{notes}

%---------------------------------------

\chapter{τό πάθος}

\fig{emotions/emotions.svg}

\begin{notes}
Descriptions of emotional state that in English would be expressed using to be+adjective (``I am happy'') are
instead expressed using verbs, e.g., ``χαίρω.''
Modern χαίρετε is a greeting, meaning to rejoice in the gospels.
ἕλεος=pity. μῆνις=anger (Iliad 1.1). χωόμενος=angry (Iliad 1.44).
φόβος, φοβέω literally refer to fleeing. ὕβρις=hubris, ὑπεροπλίη=scorn.
θαυμα=wonder, marvel, surprise. φιλία=love, friendship. ἕρως=love, erotic love.
Cognates: erotic, phobia, hubris.
\end{notes}

%---------------------------------------

\chapter{ὁ νους}

\fig{mind/mind.svg}

\begin{notes}
φρονέω=understand, intend.
\end{notes}

%---------------------------------------

\chapter{ὁ οὐρανός}

\fig{sky/sky.svg}

\begin{notes}
Cognates: helium, selenuim, asteroid, brontosaurus, anemometer, pneumatic, \emph{Eohippus}=dawn horse.
\end{notes}

%---------------------------------------

\chapter{prepositions}

\fig{prepositions/prepositions.svg}

\begin{notes}
A λέβης was a cooking pot, used as a unit of value in trade. The word occurs frequently in Homer in descriptions
of lavish gifts or treasures. NAGD = λέβης, λέβητα, λέβητος, λέβητι.
\end{notes}

%---------------------------------------

\chapter{τρόποι ἀνθρώπων}

\fig{character/character.svg}

\begin{notes}
καλή = good, beautiful, well formed. κακή = bad, lowborn, cowardly, ugly, evil. νέα = young.
γεραιός = old. αἰδοῖός = dignified. δεινός = terrible, great. ἀγαθός = highborn, brave.
λαμπρός = brilliant.
\end{notes}

\end{document}
